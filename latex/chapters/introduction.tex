\chapter{Giới thiệu đề tài}

\section{Đặt vấn đề}
Hiện nay với sự bùng nổ của thương mại điện tử,
việc mua bán và thanh toán trực tuyến đã trở nên phổ
biến rộng rãi trong cộng đồng, vì vậy lượng hàng hóa
cung và cầu mỗi ngày luôn rất lớn. Điều này rất có lợi
cho những người tiêu dùng, bởi sẽ có rất nhiều mặt hàng
với đủ mọi chủng loại, càng nhiều lựa chọn mua hàng thì giá cả càng rẻ,
những người tiêu dùng thông thái sẽ có thể mua được những sản phẩm
tốt với mức giá phải chăng. Đối với những người tiêu dùng nghi ngờ
về hình ảnh, chất lượng của các sản phẩm trực tuyến, họ hoàn toán
có thể đến các cửa hàng bán lẻ, nơi mà hàng hóa vẫn
luôn dồi dào, được nhìn tận mắt, được cầm tận tay.

Các cửa hàng bán lẻ là điểm cuối trong mạng lưới vận chuyển hàng hóa
trước khi sản phẩm đến được tay người tiêu dùng. Khi hàng hóa nhiều lên
và các phương pháp quản lý kho cũ không còn đáp ứng được thì mỗi công
ty hay tập đoàn bán lẻ đều cần giải pháp mới để quản lý hàng hóa. 

Hệ thống quản lý phân phối (Distribution Management System) là một ứng dụng
cung cấp giải pháp cho vấn đề này. Với hệ thống quản lý phân phối,
các công ty hay tập đoàn bán lẻ có thể nắm
được chính xác biến động thị trường,
hàng hóa đang ở vị trí nào trong chuỗi cung ứng, hàng tồn kho còn bao
nhiêu hay hiệu suất làm việc của các nhân viên bán hàng, …

\section{Mục tiêu và phạm vi đề tài}
Từ sự cần thiết phải chuyển đổi phương pháp quản lý phân phối này,
nhiều phần mềm nghiệp vụ đã được triển khai. Ở Việt Nam hiện nay, các ứng
dụng quản lý phân phối trực tuyến phổ biến có thể kể đến như
KiotViet, Sapo, Suno, …, các ứng dụng quản
lý không cần trực tuyến như Adaline,
BS Silver, Perfect Warehouse, … Điểm chung của các ứng dụng này là
đều cung cấp các tính năng phục vụ về quản lý kho,
nhập – xuất sản phẩm, các ứng dụng trực tuyến
thì đưa ra nhiều giải pháp hơn bao gồm cả quản lý nhân viên, quản lý lương,
thậm chí có cả kênh bán hàng riêng. KiotViet hay Sapo là những cái
tên tiêu biểu khi mà sản phẩm của hai tập đoàn này khá phổ biến.

Trong quá trình sử dụng thử phần mềm của KiotViet,
chúng tôi thấy các tính năng của
họ đưa ra là rất tốt và khá đầy đủ cho nhiều loại sản phẩm, mặt hàng.
Tuy nhiên, chúng tôi muốn bổ sung thêm tính năng quản lý tuyến bán hàng
thông qua bản đồ, quản lý hiệu quả làm việc của nhân viên
bán hàng qua thông tin check-in, cải thiện giao diện cũng
như tăng hiệu năng của ứng dụng. 

Ứng dụng quản lý phân phối của chúng tôi sẽ tập trung vào 7 tính năng chính,
đó là phân quyền động, quản lý tài khoản, quản lý sản phẩm, quản lý kho,
quản lý xuất nhập, quản lý tuyến bán hàng và quản lý nhân viên bán hàng.

\section{Định hướng giải pháp}
Go là một ngôn ngữ lập trình mới do Google phát triển,
được sinh ra để giúp ngành công nghiệp phần mềm khai thác nền
tảng đa lõi của bộ vi xử lý và hoạt động đa nhiệm tốt hơn.
Vì vậy chúng tôi sử dụng Go để viết back-end server nhằm tăng hiệu
năng xử lý yêu cầu trên một thời điểm.

Còn về phần giao diện, giao tiếp với người dùng (front-end) thì hiện
nay các Framework Javascript như ReactJS, Angular hay VueJS đang
là xu thế bởi khả năng xây dựng giao diện nhanh, bảo trì và
mở rộng code dễ dàng. Ứng dụng của chúng tôi sử dụng ReactJS, kết hợp Redux và
Material-UI để xây dựng giao diện đảm bảo thiết kế
chuẩn Material Design của Google. 

\section{Bố cục đồ án}
Phần còn lại của báo cáo đồ án tốt nghiệp được tổ
chức thành các chương như sau.

Chương 2 trình bày về cơ sở lý thuyết của đồ án. Trong đó bao gồm
các công nghệ cốt lõi đã tìm hiểu để xây dựng ứng dụng, so sánh
giữa các công nghệ được sử dụng với các công nghệ khác hiện nay.
Bên cạnh đó là các thuật toán cơ sở được sử dụng trong xử lý logic
cho cả phần front-end, back-end và cơ sở dữ liệu như Index,
Role-Based Access Control, thuật toán phân cụm. 

Chương 3 trình bày chi tiết về thiết kế hệ thống. Từ sự cần thiết
của giải pháp đã nêu ở trên, chúng tôi xác định ra những tính năng cần thiết
nhất và xây dựng các ca sử dụng xung quanh những tính năng này.
Phần này sẽ trình bày các biểu đồ ca sử dụng cho các chức năng,
biểu đồ hoạt động thể hiện cách thức tương tác với hệ thống của
người dùng. Cùng với đó là thiết kế cơ sở dữ liệu, xây dựng dữ liệu mẫu.

Chương 4 đưa ra đóng góp chính của chúng tôi trong đồ án tốt nghiệp này,
so sánh chúng với những giải pháp hiện tại, cơ hội áp dụng trong
tương lai và khả năng mở rộng của đóng góp này. Chương này cũng
đưa ra những vấn đề và hướng giải quyết của những vấn đề này
trong quá trình thực hiện đồ án. 

Chương 5 minh họa các chức năng của hệ thống, mô tả quy trình
triển khai ứng dụng lên server. Đánh giá hiệu năng hệ thống bằng
stress test và performance test. Đưa ra các vấn đề, thách thức
khi triển khai hệ thống với nhiều người sử dụng và với rất
nhiều người sử dụng.

Chương 6 là chương cuối cùng đưa ra kết luận, những vấn đề còn
chưa giải quyết được và hướng phát triển của ứng dụng trong tương lai.
