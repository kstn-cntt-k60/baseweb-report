\chapter{GIỚI THIỆU ĐỀ TÀI}

\section{Đặt vấn đề}
Hiện nay, với sự phát triển kinh tế và sự phát triển của thương mại
điện tử, lượng hàng hóa cung và cầu càng ngày càng lớn. Kéo theo lượng
hàng hóa lớn đó là vấn đề phân phối và quản lý sao cho hiệu quả. Khi hàng
hóa nhiều lên, các phương pháp quản lý kho cũ không còn đáp ứng được
thì mỗi công ty hay tập đoàn phân phối bán lẻ đều cần phải đưa ra giải pháp
mới để quản lý hàng hóa.

Hệ thống quản lý phân phối (Distribution Management System) là một ứng dụng
cung cấp giải pháp cho vấn đề này. Với hệ thống quản lý phân phối,
các công ty hay tập đoàn bán lẻ có thể nắm
được chính xác biến động thị trường,
hàng hóa đang ở vị trí nào trong chuỗi cung ứng, hàng tồn kho còn bao
nhiêu hay hiệu suất làm việc của các nhân viên bán hàng, …

Cùng với vấn đề quản lý một lượng ngày càng lớn các hàng hóa là sự phức
tạp của các hệ thống quản lý phân phối (DMS) ngày càng cao. Nhờ sự
phát triển của công nghệ trong thời gian hiện nay, đặc biệt là công nghệ
Web đã tạo ra nhiều hướng đi, giải pháp mới trong xây dựng các hệ thống
thông tin nói chung và DMS nói riêng. Tuy nhiên, với những doanh nghiệp
đã và đang sử dụng những công nghệ không còn phổ biến, việc thay đổi
công nghệ luôn là một điều khó khăn.

\section{Mục tiêu và phạm vi đề tài}
Từ sự cần thiết phải chuyển đổi phương pháp quản lý phân phối, nhiều phần
mềm nghiệp vụ đã được triển khai. Ở Việt Nam hiện nay, các ứng dụng
quản lý bán lẻ trực tuyến phổ biến có thể kể đến như KiotViet, Sapo, Suno,
... Các ứng dụng quản lý không cần trực tuyến như Adaline, BS Silver,
Perfect Warehouse, ... Các ứng dụng quản lý bản lẻ này tuy cũng có
những tính năng như quản lý sản phẩm, đơn hàng, ... tuy nhiên lại đơn giản
so với một hệ thống quản lý phân phối đầy đủ của một doanh nghiệp lớn về
phân phối hàng hóa.

Do sự phức tạp ngày càng cao của các hệ thống quản lý phân phối, tốc độ
triển khai phát triển hệ thống cũng như hiệu năng của hệ thống ngày cảm
suy giảm. Các công nghệ mới đang dần trở thành xu hướng hiện nay có tiềm
năng để giải quyết những vấn đề trên. Tuy nhiên, các doanh nghiệp đôi
khi cũng không sẵn sàng cho việc thay đổi đó.

Vì vậy, chúng tôi muốn xây dựng một ứng dụng quản lý phân phối riêng biệt,
áp dụng công nghệ phát triển và triển khai mới nhất, có đánh giá định tính
nhằm xác định và cải thiện hiệu năng hệ thống, có phương pháp xây dựng
nhằm khai thác được lợi thế của các dịch vụ Cloud. Ứng dụng quản lý
phân phối của chúng tôi sẽ tập trung vào 8 tính năng chính,
đó là phân quyền động, quản lý tài khoản, quản lý sản phẩm, quản lý kho,
quản lý xuất nhập, quản lý đơn hàng, quản lý tuyến bán
hàng và quản lý nhân viên bán hàng.

\section{Định hướng giải pháp}
Go là một ngôn ngữ lập trình mới do Google phát triển,
được sinh ra để giúp ngành công nghiệp phần mềm khai thác nền
tảng đa lõi của bộ vi xử lý và hoạt động đa nhiệm tốt hơn.
Vì vậy chúng tôi sử dụng Go để viết back-end server nhằm tăng hiệu
năng xử lý yêu cầu trên một thời điểm.

Còn về phần giao diện, giao tiếp với người dùng (front-end) thì hiện
nay các Framework Javascript như ReactJS, Angular hay VueJS đang
là xu thế bởi khả năng xây dựng giao diện nhanh, bảo trì và
mở rộng code dễ dàng. Ứng dụng của chúng tôi sử dụng ReactJS, kết hợp Redux và
Material-UI để xây dựng giao diện đảm bảo thiết kế
chuẩn Material Design của Google. 

Về cơ sở dữ liệu, chúng tôi sử dụng PostgreSQL vì đó là phần mềm mã nguồn
mở có hiệu năng và tính mở rộng cao. Đồng thời sử dụng Redis làm nơi lưu
trữ thông tin phiên làm việc (session) và dữ liệu cache.
Chúng tôi sử dụng một công nghệ container đang được
ưa chuộng là Docker và công nghệ điều phối là Kubernetes để xây dựng
và triển khai ứng dụng. Còn về dịch vụ cloud, chúng tôi sử dụng Microsoft
Azure làm nhà cung cấp dịch vụ cloud chính để ứng dụng chạy trên.

Để có thể đánh giá hiệu năng hệ thống, chúng tôi kết hợp đánh giá hiệu năng
dựa vào công cụ có trên PostgreSQL và dựa vào chương trình dòng lệnh viết
bằng Go để giả lập các truy cập từ người dùng.

\section{Bố cục đồ án}
Phần còn lại của báo cáo đồ án tốt nghiệp được tổ
chức thành các chương như sau.

Chương 2 trình bày về cơ sở lý thuyết của đồ án. Trong đó bao gồm
các công nghệ cốt lõi đã tìm hiểu để xây dựng ứng dụng, so sánh
giữa các công nghệ được sử dụng với các công nghệ khác hiện nay.
Bên cạnh đó là các thuật toán cơ sở được sử dụng trong xử lý logic
cho cả phần front-end, back-end và cơ sở dữ liệu như Index,
Role-Based Access Control, thuật toán phân cụm. 

Chương 3 trình bày chi tiết về thiết kế hệ thống. Từ sự cần thiết
của giải pháp đã nêu ở trên, chúng tôi xác định ra những tính năng cần thiết
nhất và xây dựng các ca sử dụng xung quanh những tính năng này.
Phần này sẽ trình bày các biểu đồ ca sử dụng cho các chức năng,
biểu đồ hoạt động thể hiện cách thức tương tác với hệ thống của
người dùng. Cùng với đó là thiết kế cơ sở dữ liệu, xây dựng dữ liệu mẫu.

Chương 4 đưa ra đóng góp chính của chúng tôi trong đồ án tốt nghiệp này,
so sánh chúng với những giải pháp hiện tại, cơ hội áp dụng trong
tương lai và khả năng mở rộng của đóng góp này. Chương này cũng
đưa ra những vấn đề và hướng giải quyết của những vấn đề này
trong quá trình thực hiện đồ án đồng thời 
minh họa các chức năng của hệ thống.

Chương 5 mô tả quy trình triển khai ứng dụng lên cloud.
Đánh giá hiệu năng hệ thống bằng stress test và performance test
thông qua cơ sử dữ liệu và REST API. 

Chương 6 là chương cuối cùng đưa ra kết luận, những vấn đề còn
chưa giải quyết được và hướng phát triển của ứng dụng trong tương lai.
