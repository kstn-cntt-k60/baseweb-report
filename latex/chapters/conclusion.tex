\chapter{Kết luật và hướng phát triển}
\section{Kết luận}
Qua quá trình tìm hiểu và thực hiện hệ thống quản lý phân phối,
nghiên cứu quy trình và các chức năng nghiệp vụ, thiết kế giao
diện, cơ sở dữ liệu thì phần mềm nhóm em tạo ra đã cơ bản hoàn
thiện, đáp ứng đủ những ca sử dụng như thiết kế. Phần giao diện
viết bằng ReactJS theo chuẩn Single Page Application, thiết kế theo
phong cách Material Design. 

Để trở thành một dự án khả thi trong thực tế thì ứng dụng cần phải
khắc phục nhiều hạn chế. Như đã trình bày, hiện nay KiotViet hay Sapo
là những công ty cung cấp các ứng dụng quản lý phân phối rất lớn,
việc cạnh tranh một chỗ đứng với sản phẩm của họ là hạn chế đầu tiên.
Phần giao diện còn tương đối sơ sài vì chưa có nhiều thời gian chỉnh sửa,
việc này đòi hỏi nhiều thứ như nên thiết kế trang chủ thế nào,
thiết kế logo, vị trí đầu trang và chân trang cần những thông tin
như thế nào, hiệu ứng trượt dọc, dropdown, … Ở các công ty thì họ có
kỹ sư thiết kế UX/UI, một đội photoshop, một đội dựng giao diện
và hiệu ứng, do đó đây là hạn chế thứ hai của nhóm em. Hạn chế
tiếp theo là, dù đã sử dụng webpack để nén mã nguồn, Single Page
để tối ưu tốc độ tải của trang web nhưng do dùng nhiều thư viện
vẫn chưa thực sự kiểm soát được hiệu năng của front-end. Ví dụ
như trang thegioididong.com có tốc độ tải trang vô cùng nhanh,
vì họ có nhiều giải pháp tối ưu. Ảnh của thegioididong rất nhỏ,
chỉ tầm 20-40kb, điều này cho thấy các ảnh được tối ưu rất kỹ,
họ cũng không dùng bootstrap; ngoài ra trang còn áp dụng lazy load tức
khi cuộn trang xuống mới thấy các ảnh bên dưới. Thứ hai, họ xử lý
CSS và JS đúng cách, không phải tải bất kì file CSS nào mà bỏ toàn
bộ CSS vào head (giúp thời gian render giảm từ 378ms xuống còn 225ms).
Thứ ba, họ cache mọi thứ, cụ thể các tài nguyên như ảnh, CSS, JS
được cache trong vòng 1 năm, do vậy khi load lại trang trình
duyệt không cần tải lại ảnh, CSS hay JS này nữa. Một hạn
chế nữa, ứng dụng chưa có giải pháp xử lý trong tình huống có nhiều
request lên server cùng một lúc, ví dụ như có nhiều yêu cầu lên
đơn hàng đồng thời trong khi số lượng hàng trong kho lại không đủ.

Sau khi thực hiện đồ án tốt nghiệp, bản thân em nhận thấy đã có
nhiều tiến bộ về cả kiến thức và kỹ năng. Em đã được thực hành nhiều
công nghệ mới về web front-end, web back-end, cơ sở dữ liệu;
biết được những công nghệ này là lợi thế lớn khi đi làm sau này.
Học được cách xử lý tình huống khi gặp lỗi (lỗi cú pháp lập trình,
lỗi logic, …), tỉ mỉ hơn khi thiết kế giao diện, học được cách tối ưu
hiệu năng front-end, thuật toán phân cụm dữ liệu. Về kỹ năng thì
em đã học được về kỹ năng làm việc nhóm khi mà mỗi người làm một
nhiệm vụ khác nhau, quản lý code qua các branch trên GitHub;
kỹ năng tìm kiếm tài liệu trên Google, tìm đúng từ khóa,
đúng nội dung cần tìm; … Và hơn nữa sau khi làm đồ án, em đã có
một sản phẩm thật, chạy được đáp ứng đủ yêu cầu về nghiệp vụ
cho hệ thống quản lý phân phối trong chuỗi cung ứng.

\section{Hướng phát triển của đồ án trong tương lai}
Là một phần mềm mang tính thương mại nên em rất muốn có thể
triển khai ứng dụng trong thực tế. Để làm được việc này thì trước tiên
em cần hoàn thiện phần giao diện của mình, sau đó là chỉnh sửa lại
các tính năng đã có cho dễ sử dụng hơn, đẹp hơn; khắc phục
các hạn chế đã nêu ở trên; thiết kế thêm các tính năng mới.
