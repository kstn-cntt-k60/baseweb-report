\subsection{Docker}
\subsubsection{Công nghệ container}
Công nghệ container được sinh ra có mục đích tương tự như công
nghệ máy ảo (Virtual Machine)~\cite{dockerbook}.
Điểm khác biệt lớn nhất là mỗi container
sẽ không yêu cầu phải chạy một hệ điều hành đầy đủ và riêng biệt.
Tất cả các container trên một máy chủ đều chia sẻ chung một hệ
điều hành. Chính điểm khác biệt này làm cho container sử dụng ít tài
nguyên CPU, RAM và bộ nhớ lưu trữ hơn VM rất nhiều dẫn đến tiết
kiệm chi phí cho việc chạy server, vá lỗi hệ điều hành và các
vấn đề bảo trì hệ thống khác.

Công nghệ Container hiện đại được bắt đầu từ hệ điều hành Linux.
Một số các công nghệ của nhân Linux giúp tạo nên sự phát triển của
công nghệ container bao gồm: kernel namespace, control group, union
filesystem. Dù cho các công nghệ này đã tồn tại từ lâu những để
sử dụng nó thì rất phức tạp và ngoài tầm với của phần lớn các
công ty công nghệ. Sự thay đổi bắt đầu khi Docker được tạo ra
giúp đơn giản hóa rất nhiều quá trình tạo và quản lý các container. 

Docker ban đầu được tạo nên chỉ để chạy trên môi trường Linux.
Tuy nhiên, trong một vài năm gần đây, Microsoft đã phát triển
công nghệ container trên hệ điều hành Windows 10 và Windows Server từ
đó giúp việc sử dụng Docker trên Windows tương tự như trên Linux.
Nhưng bởi vì container sẽ chia sẻ hệ điều hành với hệ điều hành
mà Docker chạy trên (được gọi là host), từ đó dẫn đến một container
dùng Linux sẽ không thể chạy trên Windows và ngược lại.

\subsubsection{Sơ lược về Docker}
Docker được bắt đầu phát triển bởi Docker, Inc vào năm 2013,
bao gồm các thành phần:
\begin{itemize}[topsep=0ex]
\item Software: gồm một chương trình chạy ngầm (daemon) gọi là dockerd,
    là chương trình quản lý các container và xử lý các container object.
    Daemon dockerd lắng nghe các yêu cầu từ bên ngoài gửi đến thông qua
    Docker Engine API. Đi kèm là một chương trình client, gọi là docker,
    cung cấp một giao diện dòng lệnh giúp tương tác với daemon dockerd.

\item Object: các Docker Object là các thực thể cấu thành lên một
    ứng dụng chạy trên Docker. Các Docker object được chia thành ba
    loại là image, container và service.
    \begin{itemize}
    \item Một Docker image là một tập các file chỉ đọc dùng làm
        template để tạo nên các container. Các image được dùng để
        lưu trữ và triển khai các ứng dụng.

    \item Một Docker container là một môi trường được đóng gói
        và chuẩn hóa để chạy ứng dụng.

    \item Một Docker service là một thực thể dùng để chạy nhiều
        container trên nhiều máy trong một cluster. Tập các máy chạy
        Docker trong một cluster gọi là một Swarm.  
    \end{itemize}
\end{itemize}

\subsubsection{Dockerfile}
Để xây dựng các image cho việc đóng gói ứng dụng, ta có thể
chạy container, cài đặt các file cần thiết và sử dụng lệnh
\textbf{docker commit} để lưu container thành image. Tuy nhiên thông thường
việc tạo image thông qua việc sử dụng Dockerfile.

Dockerfile là một file văn bản có chứa những instruction(chỉ thị) để xây
dựng một image. Trong đó có các lệnh như: chỉ định image gốc nào mà
sẽ được xây dựng lên, chỉ định lệnh nào được chạy khi bắt đầu
một container, chỉ định những file hay thư mục nào được sao
chép từ host vào container, ...
. Để xây dựng lên image bằng Dockerfile, ta sử dụng docker build.

Ví dụ trong đồ án này sử dụng Dockerfile sau để
xây dựng lên Webserver viết bằng Go:

\begin{lstlisting}[caption={Xây dựng image của back-end server},captionpos=b]
FROM golang:1.14.3-alpine3.11 as builder
WORKDIR /go/baseweb/
COPY . .
RUN go build

FROM alpine:3.11
WORKDIR /go/
COPY --from=builder /go/baseweb/baseweb .
EXPOSE 8080
CMD ["./baseweb"]
\end{lstlisting}

\subsubsection{Registry \& Docker Hub}
Registry là một kho lưu trữ các image sử dụng để tải xuống image
và tạo container từ đó. Registry có thể là một private server trong
một cluster, hay một dịch vụ trên cloud (như Azure Container
Registry) hay phổ biến và mặc định của Docker là Docker Hub.  

Để đăng nhập vào Docker Hub bằng Docker client, ta sử dụng lệnh
\textbf{docker login} và để đẩy image lên registry,
ta sử dụng \textbf{docker push}.

Docker Hub hỗ trợ cả public repository và private tương tự như GitHub. 
