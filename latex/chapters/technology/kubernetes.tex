\subsection{Kubernetes}
\subsubsection{Sơ lược về Kubernetes}
Kubernetes (viết tắt là K8s) là một phần mềm mã nguồn mở cho việc
tự động hóa quá trình triển khai (deploy), mở rộng (scale) và
quản lý các ứng dụng đã được đóng gói để chạy trên container.
Ban đầu được thiết kế bởi Google và hiện tại được quản lý bởi
Cloud Native Computing Foundation. K8s hoạt động với nhiều công cụ
quản lý container, phổ biến nhất là với Docker. Rất nhiều các
nhà cung cấp dịch vụ Cloud như AWS, Google Cloud và Azure cho phép
các ứng dụng triển khai bằng Kubernetes có thể chạy trực tiếp
từ dịch vụ Kubernete mà không phải tự thiết
lập cluster sao cho phù hợp. 

\subsubsection{Các object trong Kubernetes}
Kubernetes định nghĩa một tập các khối căn bản khi hoạt động cùng
nhau sẽ cung cấp cơ chế giúp triển khai, duy trì và mở rộng ứng
dụng dựa trên CPU, bộ nhớ hoặc là một metric bất kì.
Các object chính trong Kubernetes bao gồm:
\begin{itemize}[topsep=0ex]
\item Pod: là một tập các container mà luôn được đảm bảo sẽ chạy
trên cùng một host và cho phép chia sẻ tài
nguyên với nhau~\cite{kubernetes-book}.
Pod là đơn vị cho thuật toán phân bố container trong Kubernetes.
Mỗi pod được gán một địa chỉ IP duy nhất trong cluster. Các
container trong cùng một pod có thể tham chiếu lẫn nhau sử dụng
địa chỉ localhost, những các container ở khác pod sẽ không thể
truy cập trực tiếp với nhau thông qua localhost, khi đó phải
sử dụng địa chỉ IP mà mỗi Pod được gán. Tuy nhiên, ứng dụng
chạy trên container nằm trong pod không nên sử dụng địa
chỉ IP trực tiếp của Pod bởi vì địa chỉ IP của Pod là không tĩnh,
có thể bị thay đổi khi một Pod khởi động lại.

\item Replica Set: là một cơ chế với mục tiêu đảm bảo số lượng ổn
định của các Pod trong cluster tại bất kì thời điểm nào.
Thông thường được sử dụng để chỉ định số lượng Pod giống nhau
sẽ được chạy trong cluster.

\item Service: Một Service trong Kubernetes là cơ chế nhóm các pod
hoạt động cùng với nhau lại, cung cấp cơ chế Load Balance theo
round-robin cho phép container nằm trong Pod ở bên ngoài service
có thể truy cập các Pod ở bên trong service thông qua địa
chỉ IP của Service thay vì địa chỉ IP trực tiếp của các Pod.
Địa chỉ IP này có thể được lấy thông qua các biến môi trường hoặc
thông qua một server DNS trong cluster. Service mặc định là
public ở trong cluster. Tuy nhiên nếu muốn truy cập service từ
bên ngoài cluster ta phải thay đổi kiểu của Service thành 
\textbf{type: LoadBalancer}.
Khi đó Service sẽ chứa địa chỉ IP cho phép truy
cập từ bên ngoài cluster.

\item Deployment: là một cơ chế được xây dựng trên Replica Set
cho phép thay đổi cấu hình của các Pod hoặc của Replica Set.
Những thay đổi này được mô tả bởi người dùng bởi việc chỉ định
cấu hình mong muốn, Deployment sẽ cập nhật các Pod và các Replica
Set tương ứng để trạng thái hệ thống đạt được cấu hình mong muốn đó. 
\end{itemize}
Ngoài ra còn các object khác như ReplicationController,
StatefulSet, Volume, Endpoint, ...
