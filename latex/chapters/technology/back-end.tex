\subsection{Công nghệ back-end}
\subsubsection{Giới thiệu về ngôn ngữ Go}
Go là ngôn ngữ biên dịch kiểu tĩnh được tạo ra tại Google.
Cú pháp của Go tương tự như của ngôn ngữ C, đồng thời cũng
bị ảnh hưởng lớn bởi ngôn ngữ này. Ngôn ngữ Go cũng thông
thường được gọi với tên là Golang.
Sau đây là một số đặc điểm của ngôn ngữ Go:
\begin{itemize}[topsep=0ex]
\item Cú pháp và môi trường lập trình của Go có
    tính tương đồng với các ngôn ngữ lập trình
    kiểu động (dynamic language) như:
    \begin{itemize}[topsep=0ex]
    \item Rút gọi khai báo biến, không cần phải khai báo kiểu
        thông qua cơ chế type inference.
    \item Biên dịch nhanh.
    \item Quản lý các gói thư viện từ xa và các
        package có document truy cập online.
    \end{itemize}

\item Sử dụng cách tiếp cận đặc biệt với một số vấn đề:
    \begin{itemize}[topsep=0ex]
    \item Xây dựng sẵn trong ngôn ngữ một số cơ chế cho lập trình
        đồng bộ như: tiến trình nhẹ (light-weight process,
        hay còn gọi là goroutine), channel và câu lệnh select.
    \item Toolchain mặc định khi biên dịch sinh ra file thực
        thi sẽ không chứa phụ thuộc ngoài.
    \end{itemize}

\item Ngôn ngữ được thiết kế đủ đơn giản để lập trình
    viên có thể dễ dàng hiểu và ghi nhớ.
\end{itemize}

\subsubsection{Goroutine, Chanel và khối lệnh Select}
Ngôn ngữ Go được xây dựng sẵn trong ngôn ngữ những cơ chế hỗ trợ việc
lập trình đồng bộ. Đồng bộ ở đây không chỉ là song song ở mức CPU mà
đồng thời cả việc sử dụng Asynchronous I/O cho phép các câu lệnh như
truy cập database hay đọc gói tin từ mạng chạy trong khi tiến
trình vẫn được sử dụng cho công việc khác. Kỹ thuật này phổ biến
trong các server mà sử dụng event-based.

Trung tập của xây dựng chương trình đồng bộ trong Go là tiến trình
nhẹ (light-weight process) gọi là Goroutine. Một lời gọi hàm mà
được đặt sau từ khóa go sẽ bắt đầu hàm đó trong một Goroutine mới.
Trong đặc tả ngôn ngữ (Language Specfication) không chỉ định Goroutine
được cài đặt như nào nhưng với cài đặt hiện tại thì các Goroutine
sẽ được phân bố vào một tập nhỏ các luồng ở mức hệ điều hành,
tương tự như cơ chế phân phát tiến trình trong ngôn ngữ Erlang.
Khi chương trình Go mới bắt đầu sẽ chứa duy nhất
một goroutine gọi là main goroutine.

Chương trình viết bằng Go sẽ thường sử dụng Channel, một cơ chế để
gửi các thông điệp (Message) giữa các goroutine. Channel trong Go có thể
có chứa Buffer hoặc không. Nếu không chứa Buffer thì Channel mỗi một
thời điểm chỉ chứa tối đa một Message, các lời gọi gửi message vào
channel tiếp theo đó sẽ bị block. Nếu chứa Buffer thì kích thước
của buffer cũng bị giới hạn, các message được lưu trữ và truy xuất theo
cơ chế FIFO. Khi đọc từ một channel mà channel chưa chứa message nào
cả thì goroutine đọc từ channel đó sẽ bị block. 

Channel trong Go có chứa kiểu tĩnh, nghĩa là một channel có kiểu
là chan T sẽ chỉ có thể gửi và nhận message thuộc kiểu T.
Go có chứa cú pháp đặc biệt để tương tác với channel:
\begin{itemize}[topsep=0ex]
\item \textit{x <- ch} để đọc từ channel ch vào biến x.
\item \textit{ch <- x} để gửi message x vào channel ch.
\end{itemize}
Trong trường hợp tương tác với nhiều lệnh nhận hoặc gửi message với các
channel, chương trình Go có thể sử dụng khối lệnh select để lựa chọn
lệnh nhận/gửi đầu tiên mà kết thúc block trên các channel. Cú pháp
của khối lệnh select trong Go tương tự với switch nhưng khác
là mỗi một nhãn case trong khối lệnh select sẽ là một
lệnh tương tác với channel.

\subsubsection{Các công cụ tích hợp}
Bản cài đặt bộ công cụ của Go bao gồm nhiều các công cụ
lên quan đến xây dựng, kiểm thử và phân tích code, gồm:
\begin{itemize}[topsep=0ex]
\item \textbf{go build}, để xuất file nhị phân từ các file mã nguồn.
\item \textbf{go test}, dùng để chạy các unit test và các benchmark.
\item \textbf{go fmt}, dùng để format code.
\item \textbf{go get}, tải xuống và cài đặt các thư viện và
    các file thực thi đi kèm. 
\item \textbf{go vet}, một static analyzer cho việc tìm kiếm
    các vị trí có thể có lỗi trong code. 
\item \textbf{go run}, một shortcut cho phép biên dịch
    và chạy chương trình. 
\item \textbf{go doc}, dùng để hiển thị document của thư viện. 
\item \textbf{go mod}, dùng để quản lý module - xuất hiện từ
    phiên bản Go 1.11. 
\item \textbf{go generate}, dùng để gọi code generator.
\end{itemize}
