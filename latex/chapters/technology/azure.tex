\subsection{Azure}
\subsubsection{Sơ lược về Azure}
Microsoft Azure, thông thường được gọi là Azure, là một nền
tảng điện toán đám mây (cloud computing) được tạo bởi Microsoft
cho việc xây dựng, kiểm thử và quản lý các ứng dụng và dịch vụ
chạy trên các trung tâm dữ liệu do Microsoft quản lý. Azure
cung cấp nhiều các dịch vụ điện toán đám mây ở nhiều mức độ như
Software as a Service (SaaS), Platform as a Service (PaaS)
và Infrastructure as a Service (IaaS). 

Azure được phát hành vào năm 2010 với tên là Windows Azure sau
đó được đổi tên thành Microsoft Azure vào năm 2014. 

\subsubsection{Các dịch vụ của Azure}
Azure cung cấp rất nhiều các dịch vụ, trong đó phổ biến là:
\begin{itemize}[topsep=0ex]
\item Máy ảo (virtual machine): là một Infrastructure as a Service
(IaaS) cho phép người dùng có thể chạy bất kì một máy ảo Windows
hay Linux đi kèm với các gói phần mềm phổ biến.

\item App Service: là một Platform as a Service (PaaS) cho phép
người dùng dễ dàng xuất bản và quản lý website.

\item Các cơ sở dữ liệu như SQL Server, PostgreSQL, Redis,
MySQL, MariaDB,...

\item Container Instance: cho phép chạy và quản lý các container
Docker độc lập mà không cần cơ chế điều phối như Kubernetes.

\item Kubernetes Service: cho phép chạy và quản lý một
Kubernetes cluster.
\end{itemize}
Và nhiều các dịch vụ khác.

\subsubsection{Azure Kubernetes Service}
Azure Kubernetes Service (AKS) là một dịch vụ của Azure cho phép triển
khai một cluster Kubernetes mà không phải quan tâm đến việc cài đặt một
cluster Kubernetes ra sao. Để tương tác với AKS, ta có thể sử dụng
Azure portal - một giao diện web, hoặc sử dụng Azure CLI với
chương trình có tên là az để cài đặt và chạy chương trình dòng
lệnh kubectl (chương trình quản lý cluster Kubernetes) cho phép
quản lý AKS từ xa. 
