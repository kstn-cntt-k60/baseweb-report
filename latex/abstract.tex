\addcontentsline{toc}{chapter}{Tóm tắt}

\begin{abstract}

Trong ngày nay, xã hội ngày càng hiện đại hóa, nhu cầu con
người cũng cao lên. Cùng với đó kéo theo sự phát triển của các 
công ty, cửa hàng bán lẻ và chợ thương mại điện tử. Nhu cầu 
về quản lý hàng hóa bán lẻ cũng ngày càng tăng lên. Trên thị trường
đã có một số ứng dụng cho quản lý bản lẻ phổ biến như KiotViet hay Sapo,
tuy nhiên những sản phẩm này còn sử dụng công nghệ cũ và còn một số hạn
chế. Vì vậy em muốn xây dựng một phần mềm quản lý bán lẻ trực tuyến
cung cấp các giải pháp quản lý phân phối hàng hóa.

Trong những năm gần đây, công nghệ Web đang càng trở nên phổ biến trong
xây dựng các ứng dụng quản lý doanh nghiệp. Đồng thời nhờ sự phát triển
của trình duyệt và ngôn ngữ JavaScript đã đem đến nhiều các Framework
hỗ trợ lập trình viên xây dựng những ứng dụng phức tạp, tương tác cao,
đơn trang (Single Page Application) mà không thể hoặc khó có thể thực
hiện được dựa vào công nghệ làm Web cũ. Trong đồ án này, em đã lựa chọn
những công nghệ được coi là phổ biến được sử dụng nhiều hiện nay như
ReactJS, Redux, Redux-Saga, Material-UI, Go làm back-end, Redis,
PostgreSQL.

Đến nay, đồ án đã hoàn thành được các vấn đề đặt ra và em có mong muốn
tiếp tục duy trì và phát triển thêm tính năng cho ứng dụng đồng thời đưa
hệ thống vào chạy trong thực tế.

\end{abstract}
